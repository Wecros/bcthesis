% Author: Marek Filip 2020

\chapter{Introduction}

This text serves as example content of this template and as a recap of the most important information from regulations, it also provides additional useful information, that you will need when you write a technical report for your academic work. Check out appendix \ref{jak} before you use this template as it contains vital information on how to use it.

Even though some students only need to know and comply with the official formal requirements stated in regulations as well as typographical principles to write a good diploma thesis (bachelor's thesis is a diploma thesis too -- you get a diploma for it), it is never a~bad idea to familiarize yourself with some of the well-established procedures for writing a~technical text and make things easier for yourself. Some supervisors had prepared breakdowns of proven procedures that have lead to tens of successfully presented academic works. A~selection of the most interesting procedures available to the authors of this work at the time of writing can be found in chaptes below. If your supervisor has their own web page with recommended procedures, you can skip these chapters and follow their instructions instead. If that is not the case, you should read the respective chapters proir to consulting your supervisor about the structure and contents of your academic work.

Diploma thesis is an extensive work and the technical report should reflect it. It is not easy for everyone to sit down and simply write it. You need to know where to begin and how to progress. One of many viable approaches is to start with keywords and abstract, this helps you establish what the most important part of your work is. More on that in~chapter~\ref{abstrakt}.

Once the abstract is finished, you can start with the text of the technical report. The first thing you should do is create a structure for your work, that you'll later fill with text. Chapter \ref{struktura} provides basic information and hints on writing a technical text, that can help you avoid mistakes beginners make, create chapter titles and figure out what the approximate length of individual chapters should be. The chapter concludes with an approach that should make writing a thesis much easier.

Diploma theses in the field of information technology have a specific  structure. The introduction is followed by a chapter or chapters dealing with the summary of the current state. The next chapter should evaluate the current state and provide a solution, that will be implemented and tested. The conclusion should contain evaluated results and ideas for future development. Even though the chapter titles and their length may differ from other theses, you can always find chapters that correspond with this structure. Chapter \ref{kapitoly} deals with the contents of chapters that commonly occur in dimploma theses in the field of information technology. Most students will only use a subset of all the described chapters as not everything will be relevant for their thesis. The descriptions and hints provided help students with the inner structure and the contents of chapters as well as decide whether or they should even include given chapter. 

The final chapter of a thesis is always followed by a list of references. Citations that this list is comprised of and their respective links is the subject of chapter \ref{citace}. An inexperienced student may not perceive it that way, but the list of references is a vital part of a thesis. One of the important aspects of your reviewer's evaluation is how you work with literature. A single missing entry can lead to an F for your grade, disciplinary proceedings for plagiarism and ultimately to being expelled. There are other consequences to this as two czech ministers resigned over allegations of plagiarism in 2018. Be as thorough as possible in creating your list of references.

When you're done with the text, it is necessary to figure out what the requirements for a thesis at BUT FIT are and work the kinks out. Formal requirements that are stated in~regulations and at faculty web pages can be found in chapter \ref{formality}. This chapter also contains information about the required length of different types of academic works and other helpful information. The chapter concludes with an overview of the most common mistakes that the reviewers have to deal with and that you should avoid. The review of the formal aspect of thesis is just another important part of the reviewer's assessment.

Once you deal with the formal deficiencies, you can sumbit your thesis. Before you do so, go through the checklist in appendix \ref{checklist}. The submission of paper and electronic versions of a thesis is described in chapter \ref{odevzdani}.

Chapter \ref{zaver} contains a summary of what you can learn by reading this text, and most importantly things to keep in mind before you submit your thesis.


\chapter{What had to be studied (including assessment of the current state, 40\%)}
\begin{enumerate}
  \item Study existing trading strategies for cryptocurrencies and other instruments, including rebalance and HODL. Analyze achieved results of the studied strategies and their assumptions.
  \item Study existing simulation tools suitable for testing trading strategies.
  \item Analyze the backlog of cryptocurrency trading data provided by the supervisor and summarize observed events.
  \item Propose several (adaptive) trading strategies assuming the backlog.
  \item Implement and evaluate proposed strategies vs. traditional approaches such as HODL and rebalance.
  \item Discuss further improvements and limitations of the practical deployment.
\end{enumerate}

\subsection{Study existing trading strategies for cryptocurrencies and other instruments, including rebalance and HODL. Analyze achieved results of the studied strategies and their assumptions.}

\subsection{Study existing simulation tools suitable for testing trading strategies.}

\subsection{Analyze the backlog of cryptocurrency trading data provided by the supervisor and summarize observed events.}

\chapter{New ideas that this thesis explores (30\%)}

\chapter{Implementation and evaluation (30\%)}

\chapter{Conclusion (1 page)}
\label{zaver}

This text summarized the formal requirements for a technical report of a bachelor's thesis or a dissertation. It described the usual procedures used when writing a text of technical nature and offered additional information and independent useful hints and tips for creation of a technical report of a dissertation. It was also explained that a bachelor's thesis also a~dissertation and needs to be approached as such.

It is necessary to point out that dissertation is a unique individual work, that is developed under the supervision of an experienced expert. Regardless of what this template says, you're only obliged to comply with the official guidelines stated on the faculty web pages. You always need to consider which things in the text above are relevant for a specific dissertation and which are not. Most importantly, you should listen to your supervisor, who understands the given problem the most and is therefore able to provide the best advice that you can get.

Despite the effort, it is not possible to include all the elements needed for developing a thesis in this template and guarantee that once the text, images, literature and others are added, that everything will be alright for every single dissertation. A longer text than expected will break to two lines, en entry in list of references that the style was not tested with, and in other cases the result can be hardly satisfying. It could require a modification of the template to account for an error that occurs once in hundred projects. The final PDF and consequently the printed version needs to be thoroughly checked, don't let thoughs like \uv{this was generated by the template, therefore it must be correct} cloud your judgement. If you find errors in the template or you have suggestions on how to improve it, contact us via email at \texttt{sablona@fit.vutbr.cz} and help us improve it. Any and all comments and suggestions are welcome.

Your supervisor can help you significantly when it comes to correcting errors. However, do not expect them to read through your work the night before submission deadline. For that reason, it is necessary to have everything ready in advance and consult your supervisor as you write your dissertation. Supervisor's critical viewpoint can allow for a better result and the extra effort will have a positive effect on their evaluation of the work.


Lastly, on behalf of all the authors, I would like to wish everyone currently in development of their own dissertation and those who are getting ready to start developing it a~successful completion and presentation of their work.
