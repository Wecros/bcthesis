% Author: Marek Filip 2020

\chapter{Introduction}

This text serves as example content of this template and as a recap of the most important information from regulations, it also provides additional useful information, that you will need when you write a technical report for your academic work. Check out appendix \ref{jak} before you use this template as it contains vital information on how to use it.

Even though some students only need to know and comply with the official formal requirements stated in regulations as well as typographical principles to write a good diploma thesis (bachelor's thesis is a diploma thesis too -- you get a diploma for it), it is never a~bad idea to familiarize yourself with some of the well-established procedures for writing a~technical text and make things easier for yourself. Some supervisors had prepared breakdowns of proven procedures that have lead to tens of successfully presented academic works. A~selection of the most interesting procedures available to the authors of this work at the time of writing can be found in chaptes below. If your supervisor has their own web page with recommended procedures, you can skip these chapters and follow their instructions instead. If that is not the case, you should read the respective chapters proir to consulting your supervisor about the structure and contents of your academic work.

Diploma thesis is an extensive work and the technical report should reflect it. It is not easy for everyone to sit down and simply write it. You need to know where to begin and how to progress. One of many viable approaches is to start with keywords and abstract, this helps you establish what the most important part of your work is. More on that in~chapter~\ref{abstrakt}.

Once the abstract is finished, you can start with the text of the technical report. The first thing you should do is create a structure for your work, that you'll later fill with text. Chapter \ref{struktura} provides basic information and hints on writing a technical text, that can help you avoid mistakes beginners make, create chapter titles and figure out what the approximate length of individual chapters should be. The chapter concludes with an approach that should make writing a thesis much easier.

Diploma theses in the field of information technology have a specific  structure. The introduction is followed by a chapter or chapters dealing with the summary of the current state. The next chapter should evaluate the current state and provide a solution, that will be implemented and tested. The conclusion should contain evaluated results and ideas for future development. Even though the chapter titles and their length may differ from other theses, you can always find chapters that correspond with this structure. Chapter \ref{kapitoly} deals with the contents of chapters that commonly occur in dimploma theses in the field of information technology. Most students will only use a subset of all the described chapters as not everything will be relevant for their thesis. The descriptions and hints provided help students with the inner structure and the contents of chapters as well as decide whether or they should even include given chapter. 

The final chapter of a thesis is always followed by a list of references. Citations that this list is comprised of and their respective links is the subject of chapter \ref{citace}. An inexperienced student may not perceive it that way, but the list of references is a vital part of a thesis. One of the important aspects of your reviewer's evaluation is how you work with literature. A single missing entry can lead to an F for your grade, disciplinary proceedings for plagiarism and ultimately to being expelled. There are other consequences to this as two czech ministers resigned over allegations of plagiarism in 2018. Be as thorough as possible in creating your list of references.

When you're done with the text, it is necessary to figure out what the requirements for a thesis at BUT FIT are and work the kinks out. Formal requirements that are stated in~regulations and at faculty web pages can be found in chapter \ref{formality}. This chapter also contains information about the required length of different types of academic works and other helpful information. The chapter concludes with an overview of the most common mistakes that the reviewers have to deal with and that you should avoid. The review of the formal aspect of thesis is just another important part of the reviewer's assessment.

Once you deal with the formal deficiencies, you can sumbit your thesis. Before you do so, go through the checklist in appendix \ref{checklist}. The submission of paper and electronic versions of a thesis is described in chapter \ref{odevzdani}.

Chapter \ref{zaver} contains a summary of what you can learn by reading this text, and most importantly things to keep in mind before you submit your thesis.

\chapter{Abstract}
\label{abstrakt}

Ther should be a summary of work at most 10 lines long under the Abstract heading. Despite how short it is, a good abstract provides enough information to know what the problem is, what was the chosen solution as well as the results achieved. The purpose of abstract is to let the reader know whether or not they can find the answer to their question here. The rest of this chapter was taken from professor Herout's blog \cite{Herout}.
\bigskip

\noindent First and foremost - abstract matters. Second - It's not that hard to write one. Without further ado, let's dive into it.

\subsection*{What is the purpose of an abstract}
An abstract is used for \bf searching \rm purposes, together with the title of thesis and a list of keywords. These parts (perhaps except for the title) are not directly part of the text and it's not expected that anyone who will read your thesis actually reads them. The fact that they're reading your thesis means they're past the abstract stage. Abstract serves them well to decide \bf whether or not \rm they want to read your thesis.

When someone looks for an answer to their problem, they give the librarian or a search engine (these days) keywords that directly relate to their problem. They then receive list of theses, that could possibly offer a solution based on the match between the keywords used and keywords in the theses. A good thesis title can help the person guess which texts could have a direct relation with their problem and can get them to read your thesis.

This is where abstract is crucial. The reader reads abstracts of the theses and decides, whether or not they want to read them. It also informs the them that their filter based on a title alone is wrong.

At this point, they don't have a PDF with full text or a printed version of the thesis available. Abstracts are \bf not \rm supposed to be in the text itself, but to be available on servers aggregating scientific texts. Therefore the first rule is: an abstract needs to work on it's own -- if it contains references to literature or text (``The efficiency of a method is summarized on page 51.''), it only makes a reader less interested in the author, won't read their work or cite them.


\subsection*{When and how to write an abstract}
It makes the sense to write an abstract when the writing is done -- as a summary and real annotation of the thesis.
I however like the opposite approach -- write an abstract in the beginning. Whenever I write a scientific article, I start with a long list of keywords that are related to eachother. It's a lot more than end up as the final keywords used for indexing. It help me understand where the article is headed at all times -- what should I talk about, what needs to be in the text, what does it deal with. As soon as I'm done with keywords, I form a title and an abstract.

I consider the following four parts of an abstract especially useful -- Which problem does it solve? What solution does it offer? What are the results? What is the meaning of these results? Once all of this is clear, the text essentially writes itself. If this is unclear, how on earth can you form a coherent, meaningful sentence in the same text?


\subsection*{Recommended structure of the abstract}
An abstract of a scientific thesis can consist of four parts and be useful. Each individual part consists of two to three sentences, in some cases even a single sentence is enough.

The term ``elevator pitch'' is often used in bussiness. It is not a coincidence that its structure is similar to the recommended structure of an abstract. Realistically, an abstract should contain anything the author would say about their scientific thesis if they had at most 2 minute and could not use slides, images or text. What should they talk about then?

\paragraph{Part one -- What is the problem? What is the topic? What's the goal of the text?}
\begin{itemize}
  \item{This thesis deals with.}
  \item{The goal of this thesis is.}
  \item{My aim was.}
\end{itemize}
There is no place for fairy tales specific to wrong scientific literature: ``Our five-year-plan of~work open new and bold goals for us'', ``With the evolution of computing technology and especially the display devices, it is more important than ever \ldots'' do not belong anywhere near a good text, especially an abstract. If you can express the purpose of your text in one sentence, do it and forget about everything else. Less is always more when it comes to the abstract.

\paragraph{Part two -- How is the problem solved? Is the goal fulfilled?}
\begin{itemize}
  \item{I solved the problem using this and that.}
  \item{I used this method, this procedure and analysed this.}
  \item{The work represents an algorithm that.}
  \item{I used these tools to process data and evaluated results like this.}
  \item{The principle of our algorithm is.}
\end{itemize}

There is a new methodology in the nature of scientific text (= ``how to do something''), it needs to include a description. If the solution consists of three parts, it probably means that this part of an abstract will have three sentences, where each sentence is about a~different part of the solution. A good abstract is be honest and accurate in this section -- no ``revealing secrets'' in the text itself. Vague formulations of a solution principle in~an~abstract usually means that the authors can't write or don't have anything to write about -- neither one is good enough to waste your time.

\paragraph{Part three -- What are the results? How good is the solution?}
\begin{itemize}
  \item{It was 87,3\,\% successful.}
  \item{We created a system that.}
  \item{The solution offers these options.}
  \item{As a result, we found out that.}
\end{itemize}

Stating a specific number is not a bad habit in this part -- ``we made existing XY method five times faster''. If the contributon of your work cannot be summarized in two or three sentences, something is wrong and the entire text is probably not worth writing.

\paragraph{Part four -- Well then? What does it bring to science and the reader?}
\begin{itemize}
  \item{The contribution of this thesis is.}
  \item{The primary discovery is.}
  \item{The primary result is.}
  \item{Based on the data it is possible to.}
  \item{The results allow us to.}
\end{itemize}

When writing scientific articles, I myself struggle with the similarity of third and fourth part. Both of them speak about the results and contribution of the text. The goal of the third part is to be specific and name achieved results whereas the goal of the fourth part is to interpret their meaning and significance. I guess it's fine if these two statements merge to an extent and both parts not only don't have their own paragraph, but they sometimes even intertwine with their common sentences.

\paragraph{Part zero -- What is it about? Where are we?}
\begin{itemize}
  \item{The context is this and this.}
  \item{It deals with studies of this and that.}
  \item{We build on these recent advances in our field.}
\end{itemize}

Sometimes it is necessary to insert a short specification of context at the beginning of your abstract. It can be a great asset when it comes to obscure and esoteric field, that is off to the side of the main flow. Usually this part is not needed and sentences contained here are prime examples for pseudoscientific nonsense. It's not necessarily bad to forget that this part can even exist in an abstract. If an expert in the field shakes their head after reading an abstract: ``I have no idea what this is about.'' only then it makes sense to include this part to specify context.


\subsection*{Innovation is not ignorance}

In this text I describe a general model of a general thesis. I would like to state that this is my opinion and taste and I'm interested in alternate opinions and tastes (I really am!). Every graduate (Mgr. and Bc.) feels that their thesis is special and extraordinary. Therefore they won't follow some scheme meant for common and average theses -- i.e. the others. I see good reasons to divert from the outlined scheme and recommend some students to divert from the scheme myself every year. Indeed, every thesis is unique and extraordinary. If they weren't, there would be no reason to write them, just copy them instead. Before you divert from the standard and canonical way of organizing scientific text, put some effort into learning, understanding and tackling it. The way of scientific work, structuring scientific text or citing sources can look rigid and clumsy, but for now it is the best way mankind could come up with. If you learn it, understand it's advantages and disadvantages and innovate it, it's great and you're welcome to do so. If you choose to ignore it, you'll most likely end up with a very poor innovation.

\chapter{Conclusion}
\label{zaver}

This text summarized the formal requirements for a technical report of a bachelor's thesis or a dissertation. It described the usual procedures used when writing a text of technical nature and offered additional information and independent useful hints and tips for creation of a technical report of a dissertation. It was also explained that a bachelor's thesis also a~dissertation and needs to be approached as such.

It is necessary to point out that dissertation is a unique individual work, that is developed under the supervision of an experienced expert. Regardless of what this template says, you're only obliged to comply with the official guidelines stated on the faculty web pages. You always need to consider which things in the text above are relevant for a specific dissertation and which are not. Most importantly, you should listen to your supervisor, who understands the given problem the most and is therefore able to provide the best advice that you can get.

Despite the effort, it is not possible to include all the elements needed for developing a thesis in this template and guarantee that once the text, images, literature and others are added, that everything will be alright for every single dissertation. A longer text than expected will break to two lines, en entry in list of references that the style was not tested with, and in other cases the result can be hardly satisfying. It could require a modification of the template to account for an error that occurs once in hundred projects. The final PDF and consequently the printed version needs to be thoroughly checked, don't let thoughs like \uv{this was generated by the template, therefore it must be correct} cloud your judgement. If you find errors in the template or you have suggestions on how to improve it, contact us via email at \texttt{sablona@fit.vutbr.cz} and help us improve it. Any and all comments and suggestions are welcome.

Your supervisor can help you significantly when it comes to correcting errors. However, do not expect them to read through your work the night before submission deadline. For that reason, it is necessary to have everything ready in advance and consult your supervisor as you write your dissertation. Supervisor's critical viewpoint can allow for a better result and the extra effort will have a positive effect on their evaluation of the work.


Lastly, on behalf of all the authors, I would like to wish everyone currently in development of their own dissertation and those who are getting ready to start developing it a~successful completion and presentation of their work.
