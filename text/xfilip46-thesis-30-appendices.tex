% Author: Marek Filip 2022

% This file should be replaced with your file with an appendices (headings below are examples only)

% Placing of table of contents of the memory media here should be consulted with a supervisor

\chapter{Contents of the included storage media}
\begin{verbatim}
├── Dockerfile
├── LICENSE
├── Makefile
├── README.md
├── backtester
├── presentation
├── pyproject.toml
├── requirements-lint.txt
├── requirements.txt
├── tests
├── text
├── xfilip46-thesis-presentation.pdf
└── xfilip46-thesis.pdf
\end{verbatim}


\chapter{Manual}
Usage is defined in the standard environment of bash command line interface. All needed commands are provided by the Makefile and can be listed by running \texttt{make}.

\subsection*{Installation}
Multiple options are available for installation. Python version 3.9 should be installed on the target system.

Local install:
\begin{minted}{bash}
make installdeps
\end{minted}

Virtual environment installation:
\begin{minted}{bash}
make create-venv
. venv/bin/activate
make installdeps
\end{minted}

Containerized development:
\begin{minted}{bash}
make docker-build  # docker-compose build
\end{minted}

\subsection*{How to run the project}
There are a few ways to run the project.

Via Makefile:
\begin{minted}{bash}
make run
\end{minted}

With arguments:
\begin{minted}{bash}
python -m backtester <your_arguments>
\end{minted}

Via Docker:
\begin{minted}{bash}
make docker-up  # docker-compose up
\end{minted}

The \texttt{args.yaml} configuration file should be edited to change the backtester behaviour. After that, the \texttt{simulate()} function found in the file \texttt{simulator.py} can be edited to make other strategies and plotting available.
